\documentclass{article}
\usepackage[utf8]{inputenc}
\usepackage[utf8]{inputenc}
\usepackage[english]{babel}
\usepackage{fancyhdr}
\usepackage{wrapfig}
\usepackage{amsmath, amssymb}
\usepackage{gensymb}
\usepackage[toc,page]{appendix}

\usepackage{scrextend}
\usepackage[english]{babel}
\usepackage{blindtext}


\usepackage{algorithm}
\usepackage[noend]{algpseudocode}
\makeatletter
\def\BState{\State\hskip-\ALG@thistlm}
\makeatother


\usepackage[utf8]{inputenc}
\usepackage[misc]{ifsym}
\usepackage{amsmath,amsfonts,amssymb}
\usepackage{array,booktabs}
\usepackage{parskip}

\usepackage[T1]{fontenc}
\usepackage{libertine}
\usepackage{graphicx}
\usepackage[svgnames]{xcolor}
\usepackage{framed}

\newcommand*\openquote{\makebox(26,-22){\scalebox{5}{``}}}
\newcommand*\closequote{\makebox(26,-22){\scalebox{5}{''}}}
\colorlet{shadecolor}{white}

\makeatletter
\newif\if@right
\def\shadequote{\@righttrue\shadequote@i}
\def\shadequote@i{\begin{snugshade}\begin{quote}\openquote}
\def\endshadequote{%
  \if@right\hfill\fi\closequote\end{quote}\end{snugshade}}
\@namedef{shadequote*}{\@rightfalse\shadequote@i}
\@namedef{endshadequote*}{\endshadequote}
\makeatother

 
 \usepackage[svgnames]{xcolor}
\usepackage{listings}
\lstset{language=R,
    basicstyle=\small\ttfamily,
    stringstyle=\color{DarkGreen},
    otherkeywords={0,1,2,3,4,5,6,7,8,9},
    morekeywords={TRUE,FALSE},
    deletekeywords={data,frame,length,as,character},
    keywordstyle=\color{blue},
    commentstyle=\color{DarkGreen},
}

 \usepackage{bbm}
\usepackage{mathtools}
\pagestyle{fancy}
\fancyhf{}
\fancyhead[LE,RO]{}
\fancyhead[RE,LO]{Datathon}
\fancyfoot[CE,CO]{\leftmark}
\fancyfoot[LE,RO]{\thepage}

\title{Bayesian Approach}
\author{Citadel Datathon Final }
\date{April 2019}

\usepackage{natbib}
\usepackage{graphicx}

\usepackage[T1]{fontenc}
\usepackage{titling}
\setlength{\droptitle}{-3cm}   

\begin{document}

\maketitle

\section{Introduction}

A major infectious disease surveillance body, the Centre for Disease Control in the Untied States, currently deploy an adaptation of Serfling's method (cite here) for influenza modelling. The method uses cyclic regression to model the weekly proportion of deaths from pneumonia and influenza. Since then adaptations have incorporated indicators such as counts of patient visits for influenza like illness (ILI). However, regardless of modern modifications the methodology has a particular flaw in its assumption that observations are independent and identically distributed.

In this section we attempt to shift the methodology towards the Bayesian framework in order to provide better epidemic thresholds that are adjusted for seasonal effects. In doing so, we build prior and observation models for the number of individuals infected by influenza within a specific region. After building the models we simulate from the prior to test its likeness to reality and ensure the prior model is sufficiently grounded in reality to produce justified posterior inference. Once we are satisfied with the model we deploy Approximate Bayesian Computation (ABC) to generate approximate posterior samples and proceed to make probabilistic statements to inform policy makers. 

\section{Prior Elicitation}
We begin by outlining the prior and observation models in this setting. Whilst the systematic use of parameterised distributions is not always justifiable, when building the prior we arbitrarily restrict ourselves to a parameterised density where we can make subjective evaluations of the parameters in line with our knowledge of the world.

\subsection{Prior Model}
We build a model for the number of individuals infected by influenza in a given week for a two year period. Whilst the model is agnostic to geographical location, we focus on specifying the prior distribution in line with European influenza cycles. The parameters are $ \Theta = (X_{1:104},\mu, \theta, \alpha, \rho, \ell)$, with notation defined as $(X_{1:104}) \vcentcolon= (X_{1},...,X_{104})$. From which, we model the weekly flu process $(X_{i})_{i=1}^{104}$ over a 2 year period (for example 2018 to 2019) as a weekly mean with an autoregressive process. By considering the seasonality of infection count we use a single AR process for each winter, since these winters vary in strands of influenza active, health care spending, temperature and so on. Hence we have, $X_{t}|X_{t-1},\phi = m_{t}+y_{t}$ with $y_{t} \stackrel{}{\sim} AR(\rho,50)$ and $\phi = (\mu, \theta, \rho, \ell, \alpha)$. The mean in week $t$ is given by 
$$
m_{t} = \mu + \theta t + \alpha sin^8\Big(\frac{\pi}{52}t - \ell\pi\Big)
$$

That is to say the weekly mean is a baseline infection count $\mu$, with $\theta t$ to describe secular trend, and a suitably scaled and lagged sine function to capture seasonality. 

The prior $\pi(\Theta)$ is composed of the following: 
\begin{align*} 
& X_{1}|\phi \stackrel{}{\sim} \mathcal{N}\Big(m_{1} ,\frac{50^2}{1-\rho^2}\Big) & & \mu \stackrel{}{\sim} \mathcal{U}(0,1000) \\
& X_{27}|\phi \stackrel{}{\sim} \mathcal{N}\Big(m_{1} ,\frac{50^2}{1-\rho^2}\Big) & & \theta \stackrel{}{\sim} \mathcal{U}(0,0.5)  \\
& X_{79}|\phi \stackrel{}{\sim} \mathcal{N}\Big(m_{1} ,\frac{50^2}{1-\rho^2}\Big) & & \rho \stackrel{}{\sim} \mathcal{U}(0.6,0.9) \\
& X_{t}|X_{t-1}, \phi \stackrel{}{\sim} \mathcal{N}\Big(m_{t} + \rho(X_{t-1}-m_{t-1}), 50^2\Big) & & \ell \stackrel{}{\sim} \mathcal{U}(0.7,1) \\
&      && \alpha \stackrel{}{\sim} \mathcal{U}(3000,25000)
\end{align*}
for $t \in \{2,...,26,28,...,78,80,...,104\}$. 

Given this prior model we have the following decomposition:
\begin{align*}
 \pi(\Theta)  &=  \pi(X_{1:104}|\phi)\pi(\phi) \\
 &= \pi(X_{104}|X_{1:103},\phi)\pi(X_{1:103}|\phi)\pi(\phi) \\
 &= \pi(X_{104}|X_{103},\phi)\pi(X_{1:103}|\phi)\pi(\alpha)\pi(\rho)\pi(\ell)\pi(\theta)\pi(\mu)\\
 &= \bigg[\prod_{i=2}^{26}\pi(X_{i}|X_{i-1},\phi)\bigg]\pi(X_{1}|\phi)\bigg[\prod_{i=28}^{78}\pi(X_{i}|X_{i-1},\phi)\bigg]\pi(X_{27}|\phi)\\
 &\times \bigg[\prod_{i=80}^{104}\pi(X_{i}|X_{i-1},\phi)\bigg]\pi(X_{79}|\phi)
\pi(\alpha)\pi(\rho)\pi(\ell)\pi(\theta)\pi(\mu)
\end{align*}

\subsection{Observation Model}

We model a two year period above in order to make predictions for the second year, given observations of the first. We observe the first year of recorded counts of influenza infection with noise due to poor data collection and miss-classification of illness. That is,
$$
Y_{1:52} = X_{1:52} + (\epsilon_{i})_{i=1}^{52}
$$
where $\epsilon_{i} \stackrel{iid}{\sim} \mathcal{N}(0,1)$. Thus we have
$$
\pi(X_{1:52}|\Theta) = \prod_{i=1}^{52}\mathcal{N}(Y_{i},1)
$$.

\subsection{Simulating our prior model}

Our prior model is a reductive representation of a complex random phenomena, hence it is vital to evaluate the model for likeness to the real world to ensure our posterior inference is justified. 

We first consider 100,000 samples from the prior model in Figure XXX. This graph demonstrates likeness to real observed data for Europe over the past 5 years (CITE THE GRAPH? OR WHAT?). Additionally, the credible intervals plotted show a sufficiently large range of realisations. The mean weekly flu count is 3934 (CI: 1313,6629) which further provide reasonable fit to reality, for example, in 2018 the European weekly average was 4611 patients.

#### INSERT SYNTHETIC GRAPH ####

It is important to scrutinise the prior for informativeness with respect to quantities we are particularly interested in. In Figure XXX the approximate distribution of average and maximum counts for 100,000 samples are given. Both are satisfactory, since they fall roughly uniform across wide intervals. The weekly average of 4611 in 2018 falls in the range of high density for the average, and the European 2018 maximum of 19,074 patients infected also sits in the high density region of the approximate maximum. Both distributions reflect reality well and do not over-inform. 

#### INSERT MAX/AVG GRAPH ####

\subsubsection{A Quick Remark}

When choosing a prior it is important to consider alternatives. In this project a range of distributions for each of the parameters $(\alpha, \rho, \ell, \mu, \theta)$ were considered in order to represent different states of knowledge. We verified that the results of our analysis were not sensitive to this range of priors. For example in our choice of $\mu$, which provides a base-level for the weekly mean $m_{t}$, we considered variants of uniform, normal and triangle distributions, for example $\mathcal{N}(10000,3),\mathcal{U}(3000,25000)$ and $\text{Tri}(3000,25000,10000)$. We observed reasonable similarity between the distributions and ultimately decided to work with the uniform since it best represented our prior beliefs. 

\section{Model Choice}

We are interested in understanding whether or not our current model, $\mathcal{M}_{1}$,  is adequate. In doing so, we compare its performance with alternative models whose difference with our current model is the power of sine. That is, for alternative models $\mathcal{M}_{2}, \mathcal{M}_{3}, \mathcal{M}_{4}, \mathcal{M}_{5}$ and $\mathcal{M}_{6}$ we alter the weekly mean number of influenza positive virus as:
\begin{align*}
 \mathcal{M}_{2} &:  m_{t} = \mu + \theta t + \alpha sin^{10}\Big(\frac{\pi}{52}t - \ell\pi\Big) \\
 \mathcal{M}_{3} &:  m_{t} = \mu + \theta t + \alpha sin^{12}\Big(\frac{\pi}{52}t - \ell\pi\Big) \\ 
 \mathcal{M}_{4} &:  m_{t} = \mu + \theta t + \alpha sin^{16}\Big(\frac{\pi}{52}t - \ell\pi\Big) \\
 \mathcal{M}_{5} &:  m_{t} = \mu + \theta t + \alpha sin^{20}\Big(\frac{\pi}{52}t - \ell\pi\Big) \\ 
 \mathcal{M}_{6} &:  m_{t} = \mu + \theta t + \alpha sin^{30}\Big(\frac{\pi}{52}t - \ell\pi\Big)  
\end{align*}

Here a finite number of model comparisons is made. If one wants to consider an infinite number of models a more delicate construction of the unconditional probabilities $(p_{i} : i \in \mathbf{N})$ is required (for example adhering to notions of coherence). Assuming an equal prior weighting, we progress to consider Bayes factors. 

Bayes factors depend on estimates of the marginal likelihood for the observation in question, that is, the first year falling in line with recorded data. We make use of the following consistent estimator:

\begin{algorithm}
\caption{Naive Approximation}\label{euclid}
\begin{algorithmic}[1]
\Procedure{}{}
\For {$t \in {1,...,n}$}  $ \Theta^{t} \stackrel{}{\sim} \pi(\Theta|\mathcal{M}_{k})$ 
\EndFor
\BState \emph{Calculate}:
\State
  $\hat{p} = n^{-1}\sum_{i=1}^{n}\pi(Y_{1:52}|\Theta^{i},\mathcal{M}_{k}) $
\EndProcedure
\end{algorithmic}
\end{algorithm}

When implemented using $n=100,000$ the approximation produced unstable results despite efforts to reduce computational underflow. To assess the evidence for accepting $\mathcal{M}_{k}$, $k\neq 1$, over $\mathcal{M}_{1}$ we compute the Bayes factor for the best performing of  $\mathcal{M}_{2},...\mathcal{M}_{6}$ against $\mathcal{M}_{1}$. In 10 runs we realised a range of $(0.004,12.656)$ with the Naive approximation. However, the particular $\mathcal{M}_{k}$ with the best performance was consistently $\mathcal{M}_{1}$. For this reason we proceed with $\mathcal{M}_{1}$.


\section{Posterior sampling}

Now content with the prior model we proceed to generate approximate samples of the posterior distribution given observed European data. Whilst it would be possible to generate true posterior samples, for example by using Metropolis Hastings and assessing the quality of fit with ACFs, trace plots and checking that marginal distributions agree, we rather deploy ABC to generate approximate uncorrelated samples. 

\subsection{Approximate Bayesian Computation}

With the aim to make probabilistic statements about 2019 we deploy approximate Bayesian computation to target the posterior. In doing so, we generate samples from $\pi(\Theta|Y_{1:52})$ where $Y_{1:52}$ are given by the influenza\_activity.csv.

Below we observe the first year of some synthetic data, with samples accepted by ABC in green. These samples provide a satisfactory fit to the observed process. 

#### INSERT ABC GRAPH ####

\section{Results}

Using the posterior distribution we can inform policy makers about the probability of particular magnitude outbreaks. This information allows for improved emergency planning and resource allocation. The methodology also provides the opportunity to look at the posterior for different local areas within a country. Doing so allows medical professionals to strategically allocate their resources within a country to be best prepared for combating influenza outbreaks. 

For our prior model we observe an expected maximum number of viruses testing positive for influenza at 14,487 with a 95\% credible interval of (3882,24675). This expected maximum shifts to 19,413 in the posterior with a 95\% credible interval at (12507,22085). When the observed year does not contain an epidemic these statistics also provide more justified epidemic thresholds for the following year than those currently used by the Centre of Disease Control. 

\end{document}